% --------------------------------------------------------------
% This is all preamble stuff that you don't have to worry about.
% Head down to where it says "Start here"
% --------------------------------------------------------------
 
\documentclass[12pt]{article}
 
\usepackage[margin=1in]{geometry}
\usepackage{amsmath,amsthm,amssymb,scrextend}
\usepackage{fancyhdr}
\pagestyle{fancy}

\renewcommand{\qed}{\hfill$\blacksquare$}
\let\newproof\proof
\renewenvironment{proof}{\begin{addmargin}[1em]{0em}\begin{newproof}}{\end{newproof}\end{addmargin}\qed}
 
\newenvironment{theorem}[2][Theorem]{\begin{trivlist}
\item[\hskip \labelsep {\bfseries #1}\hskip \labelsep {\bfseries #2.}]}{\end{trivlist}}
\newenvironment{lemma}[2][Lemma]{\begin{trivlist}
\item[\hskip \labelsep {\bfseries #1}\hskip \labelsep {\bfseries #2.}]}{\end{trivlist}}
\newenvironment{problem}[2][Problem]{\begin{trivlist}
\item[\hskip \labelsep {\bfseries #1}\hskip \labelsep {\bfseries #2.}]}{\end{trivlist}}
\newenvironment{exercise}[2][Exercise]{\begin{trivlist}
\item[\hskip \labelsep {\bfseries #1}\hskip \labelsep {\bfseries #2.}]}{\end{trivlist}}
\newenvironment{reflection}[2][Reflection]{\begin{trivlist}
\item[\hskip \labelsep {\bfseries #1}\hskip \labelsep {\bfseries #2.}]}{\end{trivlist}}
\newenvironment{proposition}[2][Proposition]{\begin{trivlist}
\item[\hskip \labelsep {\bfseries #1}\hskip \labelsep {\bfseries #2.}]}{\end{trivlist}}
\newenvironment{corollary}[2][Corollary]{\begin{trivlist}
\item[\hskip \labelsep {\bfseries #1}\hskip \labelsep {\bfseries #2.}]}{\end{trivlist}}
 
\begin{document}
 
% --------------------------------------------------------------
%                         Start here
% --------------------------------------------------------------

\title{Munkres Topology Solutions} % Title
\author{Hyoungchul Kim}
\date{\today} % Date for the report
\maketitle % Inserts the title, author and date

\tableofcontents

\lhead{Topology}
\chead{Hyoungchul Kim}
\rhead{\today}

\section{Section 12: Topological Spaces}

No exercises here. Just read the text.

\section{Section 13: Basis for a Topology}

\begin{exercise}{12.1}
\end{exercise}

\begin{proof}
	WTS $A = \bigcup_x U_x$. Then the proof is done since it is arbitrary union of open sets. 

Let $x \in A.$ Then there  $\exists U_x$ s.t. $x \subset U_x \subset.$ Then this is provided easily. Conversely, Let $x \in \bigcup_{x} U_x.$ Then $x \in U_x$ for some $x$. Since we have by definition that $U_x \subset A$, this is also proved.
\end{proof}

\begin{exercise}{12.4}

\end{exercise}


\end{document}
