\documentclass[a4paper]{article}

\usepackage[utf8]{inputenc}
\usepackage[T1]{fontenc}
\usepackage[english]{babel}
\usepackage{amsmath, amssymb, amsthm, amsfonts}

\newcommand{\R}{\mathbb{R}}
\newcommand{\Z}{\mathbb{Z}}
\newcommand{\N}{\mathbb{N}}
\newcommand{\Q}{\mathbb{Q}}

\title{Math Camp}
\author{Hyoungchul Kim}


\begin{document}
\maketitle
	

\section{Day 1}

\textbf{Important notes}

\begin{itemize}
	\item Due to housing issue, I might have to miss the first day.
	\item Due to this problem, I will be manually studying for about 5 to 6 subsections using the notes given by the instructor.
\end{itemize}


\noindent \textbf{Short notes in class}

\begin{itemize}
	\item Part 1: real calculus.
	\item Part 2: linear algebra.
	\item Part 3: probability and optimization.
	\item Each part lasts two weeks.
	\item HW assignment every week but not graded.
	\item Friday: Quiz (2 hours).
	\item Only thing that matters is math camp exam: Mostly everyone passes if one takes classes.
	\item Communication done through canvas.
	\item Part 1 deals with mathematical logic, spaces (metrics, topology), correspondence.
\end{itemize}

\textbf{Some notation}\\

$\N$: Natural numbers.

$\Z$: Integer numbers.

$\Q$: Rational numbers.

$\R$: Real numbers.\\

\textbf{Binary relation}\\

Binary relation $R$ on set  $X$ is a subset of  $X\times X$.

\begin{enumerate}
	\item $R$ is reflexive if:  $\forall x \in X, \, xRx$
	\item $R$ is symmetric if:  $\forall x, y \in X, \, xRy \implies yRx$
	\item $R$ is transitive if:  $\forall x, y \in X, \, xRy, yRz \implies xRz$
\end{enumerate}

\noindent \textbf{Def:} A binary relation is an equivalence relation if it is reflexive, symmetric and transitive. We denote this as $x\sim y.$\\

\noindent \textbf{Def:} $R$ is antisymmetric if  $\forall x, y \in  X, xRy \text{ and }  yRx \implies x=y.$\\

\noindent \textbf{Def:} $R$ is partial order on $X$ if it is reflexive, antisymmetric and transitive.\\ 

\noindent \textbf{Def:} $x, y \in X$ are comparable if $xRy \text{ or } yRx.$\\

\noindent \textbf{Def:} A partial order $R$ is a total order if every two elements are comparable.\\

\noindent \textbf{Def:} $f: X \to Y$ is an injection if $\forall a, b \in X, a \neq b \implies f(a)\neq f(b).$\\ 

\noindent \textbf{Def:} $f: X \to Y$ is an surjection if $Y=f(X)=\{ y \in Y | \exists x \in X: f(x)=y\}.$\\

\noindent \textbf{Def:} $f: X \to Y$ is called bijection if it is both an injection and surjection.\\ 

\noindent \textbf{Fact:} $f: X \to Y$ is an injection if $\forall y \in Y$, $f^{-1}\left( \{y\} \right) $ has no more than one element.\\ 

\noindent \textbf{Fact:} $f: X \to Y$ is an surjection if $\forall y \in Y$, $f^{-1}\left( \{y\} \right) $ has at least one element.\\ 

\noindent \textbf{Fact:} $f: X \to Y$ is an bijection if $\forall y \in Y$, \#$f^{-1}\left( \{y\} \right)=1.$\\ 

\noindent \textbf{Fact:} $f$ have inverse $\iff$ $f$ is a bijection.
\begin{proof}
	i. $f^{-1}$ exists $\implies$ $f$ is a bijection: $f^{-1}\left( \{y\} \right) &= \{ f^{-1}\left( y \right) \} \\$. Assume not $\implies \exists a \in f^{-1}\left( \{y\} \right) \, s.t. a \neq f^{-1}\left( y \right).$ But $f(a)=y \implies f^{-1}\left( f(a) \right) = f^{-1}\left( y \right).$

	ii. $f$ is a bijection $\implies$ $f^{-1}$ exists. Construct $g$ s.t. $g(y)$ is the only element of $f^{-1}\left( \{y\} \right)$. $f \cdot g: f(g(y)) = y; \, g \cdot f: g(f(x))=y$.
\end{proof}\\

\noindent \textbf{Fact:} Bijection between two finite sets exists only if $\#A &= \#B.$ \\

\noindent \textbf{Def:} $A$ and $B$ have equal cardinality if there is a bijection $f: A \to B$.\\

\noindent \textbf{Fact:} Having the same cardinality is an equivalence relation.
\begin{proof}
	i. reflexivity: $id_{A}$ is a bijection.

	ii. symmetry: $f: A \to B$ is a bijection $\implies$ $f^{-1}$ is also a bijection.

	iii. transitivity: $f:A \to B$ and $g:B \to C$ is bijection, then $g \cdot f$ is also bijection.
\end{proof}

\noindent \textbf{Def:} $A$ is denumerable if $\# A = \#\N$. (countably infinite) $A$ is countable if it is finite or denumerable. $A$ is uncountable if it is not countable.\\

\noindent \textbf{Ex:} Show $\N \times \N$ is denumerable.
\begin{proof}
	(0,0), (0,1) \ldots\\
	\indent \indent (1,0), (1,1) \ldots\\
	Just go diagonal. Then it will get you a mapping that is bijection.\\

	$f(m,n) = \frac{(m+n)(m+n+1)}{2}+m$.
\end{proof}

\noindent \textbf{Corr:} $\forall k > 0$, $\N^{k}$ is denumerable.
\begin{proof}
	later.
\end{proof}

\noindent \textbf{Prob:} Let $A \subset B$. $A$ is infinite and $B$ is denumerable. Then $A$ is denumerable.
\begin{proof}
	$f: \N \to B$. $f^{-1}(A) \subset \N$ Later solve it. 
\end{proof}

\noindent \textbf{Corr:} Let $f: A \to B$ be a surjection. If $A$ is denumerable and $B$ is infinite, then $B$ is denumerable.
\begin{proof}
	Later.
\end{proof}

\noindent \textbf{Def:} We say $\# A \le \#B$ if there is an injection $f: A \to B$. If there is no bijection from $A$ to $B$, we can say that the cardinality of $A$ is strictly smaller than that of $B$.\\

\noindent \textbf{Prob:} Prove that $\{0,1\}^{\N}$ is uncountable.
\begin{proof}
	Later. But just use the fact that if denumerable, its element can be written as a sequence.
\end{proof}

\noindent \textbf{Def:} The set of all subsets of $A$ is called power set of $A$ and is noted as $2^{A}$.\\

\noindent \textbf{Prob:} If $A$ is finite $\implies \# 2^{A} = 2^{\# A}$.
\begin{proof}
	Later.
\end{proof}

\noindent \textbf{Theorem} For every $A, \#2^{A} > \#A$.
\begin{proof}
	Injection: $x \to \{x\}$ from $A$ to $2^{A}$.

	Suppose there is a bijection $f: A \to  2^{A}$. $B = \{x \in A | x not\in f(x)\}=2^{A}. \implies \exists y \in A s.t. f(y)=B.$ 

	1. $y \in B \implies y not \in f(y)=B$ by def of $B$.

	2. $y not \in B \implies y \in f(y) =B$ by def of $B$.

	This is contradiction. Thus done.
\end{proof}

\noindent \textbf{Prob:} $\# \R = \# 2^{\N}$.
\begin{proof}
	We show that $[0,1] = 2^{\N}$ in cardinality.
\end{proof}

\noindent \textbf{Prob:} $\forall n \in \N, \R^{n} \sim \R$.
\begin{proof}
	Later.
\end{proof}

\noindent \textbf{Theorem} If $\# A \le \# B$ and vice versa, then the cardinality for both set is same.\\

\noindent \textbf{Some Exercises}\\

\noindent Exercise 1.3.11: Why such partition exists only for equivalence relation? Are all three properties required for existence of such partition?

\begin{proof}
	car
\end{proof}

\noindent Exercise 1.3.20: Let's define a binary relation $R$ on $\N$ in a following way: $xR y$ if $y$ is divisible by $x$. Is this relation a partial order, a total order or it is not order at all?

\begin{proof}
	car
\end{proof}

\noindent Exercise 1.4.10: Define a bijection in a similar way (using preimages). 

\begin{proof}
	car
\end{proof}

\noindent Exercise 1.4.11: Show that $f: X \to Y$ and $g: Y \to Z$ both are bijections, then $g \cdot f: X \to Z$ is a bijection too.

\begin{proof}
	car
\end{proof}

\noindent Exercise 1.4.15: $f: X \to Y$  has an inverse if and only if it is a bijection.

\begin{proof}
	car
\end{proof}


\section{Day 2}

\textbf{PS1 review}\\

PS 1.
\begin{proof}
	This is false. Let $p(x,y) \implies x+y=5$.
\end{proof}

PS 2.
\begin{proof}
	Obvious.
\end{proof}

PS 3.
\begin{proof}
	(i) $xRy \iff x \neq y.$

	(ii) $xRy \iff x < y+1.$

	(iii) $xRy \iff x>y.$
\end{proof}

PS 4.
\begin{proof}
	Obvious.
\end{proof}

PS extra. Set of all ftns on $\R$ $fRg$ if $\forall x \in \R \, f(x) \le g(x).$ Is it a total order, partial order or not an order?
\begin{proof}
	(1) reflexive: $fRf$ 

	(2) transitive: $fRg \, gRh \impliesfRh.$

	(3) antisymmetric.
\end{proof}

PS 5.
\begin{proof}
	No.
\end{proof}

PS 6.
\begin{proof}
	Check my notes in github.	
\end{proof}

PS 7. 
\begin{proof}
	We know that $\N \times \N \sim \N$. We just need to show there is bijection between $\Q \times \N$ and $\N \times \N$. We use the fact there there is bijection from rational number to natural number.
\end{proof}

PS 8.
\begin{proof}
	(iii) Try to use that $\left( 0, 1 \right) \times \N \to \R$ is injection and there is bijection from $\left( 0,1 \right) $ to real number.

	(iv) set of all finite seq of $\N = \cup_{n \in \N} \{ \text{all seq of length n}\}$.
\end{proof}

PS 9.
\begin{proof}
	Later.
\end{proof}

PS 10.
\begin{proof}
	$X$ is set of a possible bijections from one countable set to another.

	(i) Sps $A$ and $B$ have different cardinality $\implies$ zero.

	(ii) If they have same cardinality $m$ : $m!$.

	(iii) If they have same cardinality to  $\N$ : continuum.

	Let $X$ be set of all bijections from natural number to natural number.

	Need to show that $f: 2^{\N} \to X$ and vice versa is injection.
\end{proof}

PS 11.
\begin{proof}
	No. 

	i. $A$ is finite. Then $2^{A}$ is finite.

	ii. If $A$ is infinite. $A\sim \N \implies \# 2^{A} > \N$
\end{proof}

PS 12.
\begin{proof}
	i. $X$ is finite. Then $X = \{0, 1, \ldots, n\}$.
	Later. Just remember to move them or shift them.
\end{proof}

\textbf{Short notes in class}\\

\textbf{Def:} Vector space over $R$ is a nonempty set $V$ on which two binary operations (scalar multiplication and vector addition is defined). It has following properties:
\begin{enumerate}
	\item $u + (v+w)=(u+v)+w.$
	\item $u+w=w+u.$
	\item $\exists 0 \in V s.t.\, u+0=u.$ 
	\item $\exists -u \in V s.t. \, u+(-u)=0.$
	\item $\lambda (\mu v) = (\lambda \mu) v.$
	\item  $1 \cdot u = u.$
	\item $\left( \lambda + \mu \right) u = \lambda u + \mu u.$
	\item $\lambda (u+v) = \lambda u + \lambda v.$
\end{enumerate}

\textbf{Def:} $V$ is a v.s. Then norm can be defined as $ \mid  \mid  \cdot  \mid  \mid : V \to \R_+$.
\begin{enumerate}
	\item $ \mid  \mid u  \mid  \mid = 0 \iff v=0.$
	\item  $\mid \alpha v  \mid  \mid =  \mid \alpha  \mid  \cdot \mid  \mid v  \mid  \mid .$ 
	\item $ \mid  \mid u+v  \mid  \mid  \le   \mid  \mid  v \mid  \mid +  \mid  \mid u  \mid  \mid .$ 
\end{enumerate}

\textbf{Ex.} Let $X = \left( x_1, \ldots , x_{n} \right) \in \R^{n}.$ We can define norm as $ \mid  \mid x \mid  \mid = \sqrt{\sum_{i=1}^{n} X_{i}^{2}}.$\\

\textbf{Note.} Remember that there are many ways to construct norms.\\

\textbf{Def:} A metric on a set $X$ is a function 
\[
d: X \times X \to \R_+, \forall x,y \in X
.\] 
\begin{enumerate}
	\item $d(x,y)=d(y,x).$
	\item $d(x,y)=0 \iff x=y.$
	\item $d(x,z) \le  d(x,y) + d(y,z).$
\end{enumerate}


\textbf{Def:} A metric on a set $X$ is a function
\[
d: X \times X \to \R_+
.\] 

s.t. $\forall x, y \in X,$
\begin{enumerate}
	\item $d(x,y) = d(y,x).$
        \item $d(x,y)=0 \iff x=y.$
	\item $d(x,y) \le d(x,y) + d(y,z).$
\end{enumerate}

\textbf{Def:} A pair $(X,d)$ is called a metric space.\\

\textbf{Ex:} Trivial metric is  $d(x,y)$ where it is 0 if $x=y$ and 1 otherwise.\\

\textbf{Ex:} Normed vector space  $(V,  \mid  \mid \cdot  \mid  \mid )$. In this case, $d(x,y) =  \mid  \mid x-y \mid  \mid $ induces a metric on it.\\

\textbf{Def:} A sequence is basically a function $f: \N \to X$ and we write it as $(x_n) \in X.$\\

\textbf{Def:} A cauchy sequence is a sequence such that: $\forall \varepsilon >0, \exists N, s.t. \, \forall n, m > N, d(x_n, x_m)< \varepsilon.$\\

\textbf{Def:} A sequence $(x_n)$ converges to $x$ if 
\[
\forall \varepsilon > 0, \exists N: \forall i > N, d(x_n, x) < \varepsilon
.\] 

We denote this as $X_n \to x; \, \lim_{n \to \infty}x_n = x.$\\

\textbf{Note:} The convergence is unique. Assume it is not unique and the sequence converge to x and y. Then take $\varepsilon = \frac{d(x,y)}{3}$. There will be some $N$ large enough for some $i$ s.t. $d(x_i, x) < \varepsilon, d(x_i, y) < \varepsilon).$ This implies that  $d(x,y) < 2\varepsilon.$ But this means  $d(x,y) < 2 \varepsilon = \frac{2}{3} d(x,y) \implies \text{contradiction}.$\\

\textbf{Prop:} Every converging sequence is a cauchy sequence. But note that vice versa does not hold generally. Think about $x_n = \frac{1}{n}$ in (0,1).\\

\textbf{Theorem} In $\R$, every cauchy seq. converges.\\

\textbf{Def:} $(X,d)$ is complete if any cauchy seq. converge (e.g. $\R, \R^{n}$ with euclidean metric).\\

\textbf{Ex:} $\Q$ is not complete space. Take any seq. converging to square 2.\\


\textbf{Prop:} $C\left( [a,b] \right) $ with sup metric is a complete space.
\begin{proof}
	Let $f_n$ be a cauchy seq. For $\forall x \in [a,b],  \mid f_n (x) - f_m (x)  \mid  \le  d(f_n, f_m) = \text{max}  \mid f_n (x) - f_m (x)  \mid . \implies \forall x \in [a,b], f_n$ is a cauchy seq. 

	Define $f$ s.t. $\forall x \in [a,b], f(x) = \lim_{n \to \infty} f_n (X) \implies f_n \to f\ldots..$
\end{proof}

\textbf{Def:} Mapping $A:X \to X$ is a contraction if $\exists \lambda \in [0,1)$ s.t. $\forall x, y \in X$ $d\left( A(x), A(y) \right) \le  \lambda \cdot d(x,y)$.\\

\textbf{Def:} $x \in X$ is a fixed point of $A: X \to X$ if $A(x)=x.$\\

\textbf{Theorem (Banach fixed-point thm)} A is a contraction on complete  $(X,d)$. Then $A$ has unqiue fixed point $x^{*} \in X.$ \\


\textbf{Bellman eq.} $V(x) = max_a \left( u(a,x) + \beta V(x') \right) $, $X' = g(a,x).$  $A:f \to max_a \left( u(a,x) + \beta f(x') \right) $.


\section{Day 3}

This part has no notes because I had to miss class.

\section{Day 4}

\textbf{Short notes in class}\\

First week quiz will be (i) everything up to compacts. There will be (2) several different problems (solve what you can, skip if you can't solve). (2) You can either take this offline here at 10am tomorrow or online (quiz will posted around 9am online). Just submit before 6pm Saturday.\\

\textbf{Topology}\\

Consider some set $X$.\\

\textbf{Def: Topology}. Topology $\tau$ on $X$ is collection of subsets of $X$ s.t.
\begin{enumerate}
	\item  $\emptyset, X \in \tau$ 
	\item Closed under arbitrary union.
	\item Closed under finite intersection.
\end{enumerate}

We will call $\left( X, \tau \right) $ a topological space.\\

\textbf{Fact} If $A \subset X$ is open, its complement is closed.\\

\textbf{Fact} $\tau = 2^{x}$ is discrete topology.\\

\textbf{Def:} Cofinite topology. $U$ is open if $X - U  $ is finite.\\ 

\textbf{Def:} Any open set $u$ that contains $x$ is an open neighborhood of $x$.\\

\textbf{Def:} A topological space is Hausdorff if $\forall x, y \in X (x \neq y) $ $\exists $ open neighborhoods $U$ and $V$ of $x$ and $y$ s.t. $U \cap V = \emptyset.$\\

\textbf{Def:} Topology generated by metric is Hausdorff.
\begin{align*}
	\forall &x,y \in X (x\neq y).\\
		&d(x,y) > 0\\
	\implies &U_{\frac{d(x,y)}{3}} (x) \cap U_{\frac{d(x,y)}{3}} (y) = \emptyset.
\end{align*}

\textbf{Def:}  $f:X \to Y$ is continuous at $x \in X$ if for any open set $V \in Y$, there exists an open neighborhood $U$, $x \in U$ s.t. $f(U) \subset V$.\\

\textbf{Prop:} A mapping is continuous $\iff$ $f^{-1}(V)$ is open in $X$ for any open $V \subset Y$.\\







\end{document}
