\documentclass[a4paper]{article}

\usepackage[utf8]{inputenc}
\usepackage[T1]{fontenc}
\usepackage[english]{babel}
\usepackage{amsmath, amssymb, amsthm, amsfonts}

\newcommand{\R}{\mathbb{R}}
\newcommand{\Z}{\mathbb{Z}}
\newcommand{\N}{\mathbb{N}}
\newcommand{\Q}{\mathbb{Q}}

\title{Math Camp}
\author{Hyoungchul Kim}


\begin{document}
\maketitle
	

\section{Day 1}

\textbf{Important notes}

\begin{itemize}
	\item Due to housing issue, I might have to miss the first day.
	\item Due to this problem, I will be manually studying for about 5 to 6 subsections using the notes given by the instructor.
\end{itemize}


\noindent \textbf{Short notes in class}

\begin{itemize}
	\item Part 1: real calculus.
	\item Part 2: linear algebra.
	\item Part 3: probability and optimization.
	\item Each part lasts two weeks.
	\item HW assignment every week but not graded.
	\item Friday: Quiz (2 hours).
	\item Only thing that matters is math camp exam: Mostly everyone passes if one takes classes.
	\item Communication done through canvas.
	\item Part 1 deals with mathematical logic, spaces (metrics, topology), correspondence.
\end{itemize}

\textbf{Some notation}\\

$\N$: Natural numbers.

$\Z$: Integer numbers.

$\Q$: Rational numbers.

$\R$: Real numbers.\\

\textbf{Binary relation}\\

Binary relation $R$ on set  $X$ is a subset of  $X\times X$.

\begin{enumerate}
	\item $R$ is reflexive if:  $\forall x \in X, \, xRx$
	\item $R$ is symmetric if:  $\forall x, y \in X, \, xRy \implies yRx$
	\item $R$ is transitive if:  $\forall x, y \in X, \, xRy, yRz \implies xRz$
\end{enumerate}

\noindent \textbf{Def:} A binary relation is an equivalence relation if it is reflexive, symmetric and transitive. We denote this as $x\sim y.$\\

\noindent \textbf{Def:} $R$ is antisymmetric if  $\forall x, y \in  X, xRy \text{ and }  yRx \implies x=y.$\\

\noindent \textbf{Def:} $R$ is partial order on $X$ if it is reflexive, antisymmetric and transitive.\\ 

\noindent \textbf{Def:} $x, y \in X$ are comparable if $xRy \text{ or } yRx.$\\

\noindent \textbf{Def:} A partial order $R$ is a total order if every two elements are comparable.\\

\noindent \textbf{Def:} $f: X \to Y$ is an injection if $\forall a, b \in X, a \neq b \implies f(a)\neq f(b).$\\ 

\noindent \textbf{Def:} $f: X \to Y$ is an surjection if $Y=f(X)=\{ y \in Y | \exists x \in X: f(x)=y\}.$\\

\noindent \textbf{Def:} $f: X \to Y$ is called bijection if it is both an injection and surjection.\\ 

\noindent \textbf{Fact:} $f: X \to Y$ is an injection if $\forall y \in Y$, $f^{-1}\left( \{y\} \right) $ has no more than one element.\\ 

\noindent \textbf{Fact:} $f: X \to Y$ is an surjection if $\forall y \in Y$, $f^{-1}\left( \{y\} \right) $ has at least one element.\\ 

\noindent \textbf{Fact:} $f: X \to Y$ is an bijection if $\forall y \in Y$, \#$f^{-1}\left( \{y\} \right)=1.$\\ 

\noindent \textbf{Fact:} $f$ have inverse $\iff$ $f$ is a bijection.
\begin{proof}
	i. $f^{-1}$ exists $\implies$ $f$ is a bijection: $f^{-1}\left( \{y\} \right) &= \{ f^{-1}\left( y \right) \} \\$. Assume not $\implies \exists a \in f^{-1}\left( \{y\} \right) \, s.t. a \neq f^{-1}\left( y \right).$ But $f(a)=y \implies f^{-1}\left( f(a) \right) = f^{-1}\left( y \right).$

	ii. $f$ is a bijection $\implies$ $f^{-1}$ exists. Construct $g$ s.t. $g(y)$ is the only element of $f^{-1}\left( \{y\} \right)$. $f \cdot g: f(g(y)) = y; \, g \cdot f: g(f(x))=y$.
\end{proof}\\

\noindent \textbf{Fact:} Bijection between two finite sets exists only if $\#A &= \#B.$ \\

\noindent \textbf{Def:} $A$ and $B$ have equal cardinality if there is a bijection $f: A \to B$.\\

\noindent \textbf{Fact:} Having the same cardinality is an equivalence relation.
\begin{proof}
	i. reflexivity: $id_{A}$ is a bijection.

	ii. symmetry: $f: A \to B$ is a bijection $\implies$ $f^{-1}$ is also a bijection.

	iii. transitivity: $f:A \to B$ and $g:B \to C$ is bijection, then $g \cdot f$ is also bijection.
\end{proof}

\noindent \textbf{Def:} $A$ is denumerable if $\# A = \#\N$. (countably infinite) $A$ is countable if it is finite or denumerable. $A$ is uncountable if it is not countable.\\

\noindent \textbf{Ex:} Show $\N \times \N$ is denumerable.
\begin{proof}
	(0,0), (0,1) \ldots\\
	\indent \indent (1,0), (1,1) \ldots\\
	Just go diagonal. Then it will get you a mapping that is bijection.\\

	$f(m,n) = \frac{(m+n)(m+n+1)}{2}+m$.
\end{proof}

\noindent \textbf{Corr:} $\forall k > 0$, $\N^{k}$ is denumerable.
\begin{proof}
	later.
\end{proof}

\noindent \textbf{Prob:} Let $A \subset B$. $A$ is infinite and $B$ is denumerable. Then $A$ is denumerable.
\begin{proof}
	$f: \N \to B$. $f^{-1}(A) \subset \N$ Later solve it. 
\end{proof}

\noindent \textbf{Corr:} Let $f: A \to B$ be a surjection. If $A$ is denumerable and $B$ is infinite, then $B$ is denumerable.
\begin{proof}
	Later.
\end{proof}

\noindent \textbf{Def:} We say $\# A \le \#B$ if there is an injection $f: A \to B$. If there is no bijection from $A$ to $B$, we can say that the cardinality of $A$ is strictly smaller than that of $B$.\\

\noindent \textbf{Prob:} Prove that $\{0,1\}^{\N}$ is uncountable.
\begin{proof}
	Later. But just use the fact that if denumerable, its element can be written as a sequence.
\end{proof}

\noindent \textbf{Def:} The set of all subsets of $A$ is called power set of $A$ and is noted as $2^{A}$.\\

\noindent \textbf{Prob:} If $A$ is finite $\implies \# 2^{A} = 2^{\# A}$.
\begin{proof}
	Later.
\end{proof}

\noindent \textbf{Theorem} For every $A, \#2^{A} > \#A$.
\begin{proof}
	Injection: $x \to \{x\}$ from $A$ to $2^{A}$.

	Suppose there is a bijection $f: A \to  2^{A}$. $B = \{x \in A | x not\in f(x)\}=2^{A}. \implies \exists y \in A s.t. f(y)=B.$ 

	1. $y \in B \implies y not \in f(y)=B$ by def of $B$.

	2. $y not \in B \implies y \in f(y) =B$ by def of $B$.

	This is contradiction. Thus done.
\end{proof}

\noindent \textbf{Prob:} $\# \R = \# 2^{\N}$.
\begin{proof}
	We show that $[0,1] = 2^{\N}$ in cardinality.
\end{proof}

\noindent \textbf{Prob:} $\forall n \in \N, \R^{n} \sim \R$.
\begin{proof}
	Later.
\end{proof}

\noindent \textbf{Theorem} If $\# A \le \# B$ and vice versa, then the cardinality for both set is same.\\

\noindent \textbf{Some Exercises}\\

\noindent Exercise 1.3.11: Why such partition exists only for equivalence relation? Are all three properties required for existence of such partition?

\begin{proof}
	car
\end{proof}

\noindent Exercise 1.3.20: Let's define a binary relation $R$ on $\N$ in a following way: $xR y$ if $y$ is divisible by $x$. Is this relation a partial order, a total order or it is not order at all?

\begin{proof}
	car
\end{proof}

\noindent Exercise 1.4.10: Define a bijection in a similar way (using preimages). 

\begin{proof}
	car
\end{proof}

\noindent Exercise 1.4.11: Show that $f: X \to Y$ and $g: Y \to Z$ both are bijections, then $g \cdot f: X \to Z$ is a bijection too.

\begin{proof}
	car
\end{proof}

\noindent Exercise 1.4.15: $f: X \to Y$  has an inverse if and only if it is a bijection.

\begin{proof}
	car
\end{proof}









\end{document}
